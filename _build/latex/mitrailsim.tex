%% Generated by Sphinx.
\def\sphinxdocclass{report}
\documentclass[letterpaper,10pt,english]{sphinxmanual}
\ifdefined\pdfpxdimen
   \let\sphinxpxdimen\pdfpxdimen\else\newdimen\sphinxpxdimen
\fi \sphinxpxdimen=.75bp\relax
\ifdefined\pdfimageresolution
    \pdfimageresolution= \numexpr \dimexpr1in\relax/\sphinxpxdimen\relax
\fi
%% let collapsible pdf bookmarks panel have high depth per default
\PassOptionsToPackage{bookmarksdepth=5}{hyperref}

\PassOptionsToPackage{booktabs}{sphinx}
\PassOptionsToPackage{colorrows}{sphinx}

\PassOptionsToPackage{warn}{textcomp}
\usepackage[utf8]{inputenc}
\ifdefined\DeclareUnicodeCharacter
% support both utf8 and utf8x syntaxes
  \ifdefined\DeclareUnicodeCharacterAsOptional
    \def\sphinxDUC#1{\DeclareUnicodeCharacter{"#1}}
  \else
    \let\sphinxDUC\DeclareUnicodeCharacter
  \fi
  \sphinxDUC{00A0}{\nobreakspace}
  \sphinxDUC{2500}{\sphinxunichar{2500}}
  \sphinxDUC{2502}{\sphinxunichar{2502}}
  \sphinxDUC{2514}{\sphinxunichar{2514}}
  \sphinxDUC{251C}{\sphinxunichar{251C}}
  \sphinxDUC{2572}{\textbackslash}
\fi
\usepackage{cmap}
\usepackage[T1]{fontenc}
\usepackage{amsmath,amssymb,amstext}
\usepackage{babel}



\usepackage{tgtermes}
\usepackage{tgheros}
\renewcommand{\ttdefault}{txtt}



\usepackage[Bjarne]{fncychap}
\usepackage{sphinx}

\fvset{fontsize=auto}
\usepackage{geometry}


% Include hyperref last.
\usepackage{hyperref}
% Fix anchor placement for figures with captions.
\usepackage{hypcap}% it must be loaded after hyperref.
% Set up styles of URL: it should be placed after hyperref.
\urlstyle{same}

\addto\captionsenglish{\renewcommand{\contentsname}{Contents:}}

\usepackage{sphinxmessages}
\setcounter{tocdepth}{1}



\title{MIT RailSim}
\date{Apr 21, 2024}
\release{}
\author{Mojtaba Yousefi, Jonah Stadtmauer}
\newcommand{\sphinxlogo}{\vbox{}}
\renewcommand{\releasename}{}
\makeindex
\begin{document}

\ifdefined\shorthandoff
  \ifnum\catcode`\=\string=\active\shorthandoff{=}\fi
  \ifnum\catcode`\"=\active\shorthandoff{"}\fi
\fi

\pagestyle{empty}
\sphinxmaketitle
\pagestyle{plain}
\sphinxtableofcontents
\pagestyle{normal}
\phantomsection\label{\detokenize{index::doc}}


\sphinxstepscope


\chapter{Introduction}
\label{\detokenize{introduction:introduction}}\label{\detokenize{introduction::doc}}

\section{Project Overview}
\label{\detokenize{introduction:project-overview}}
\sphinxAtStartPar
MIT RailSim is a sophisticated simulation tool developed upon decades of research (Koutsopoulos and Wang {[}\hyperlink{cite.introduction:id7}{KW07}{]} and Zhou {[}\hyperlink{cite.introduction:id10}{Zho22}{]}) and development to address the operational challenges faced by heavy rail systems in major metropolitan areas, particularly during peak periods when demand is highest. This tool provide microscopic simulation model for heavy rail systems, enabling detailed analysis of train behaviors, signaling systems (fixed\sphinxhyphen{}block and moving block), and interactions between trains and passengers. By employing MIT RailSim, reseachers can evaluate various operating strategies such as skip\sphinxhyphen{}stop, station consolidation, and dwell control to mitigate capacity bottlenecks.

\sphinxAtStartPar
The simulation framework is built using Python in a unix environment, ensuring robust performance and flexibility. The model’s accuracy and reliability can be further enhanced through a calibration process using data from Operational Control Systems {[}\hyperlink{cite.introduction:id8}{WK11}{]}. Numerous visualization tools allow users to analyze simulation outputs.

\sphinxAtStartPar
Numerous case studies (see Zhou \sphinxstyleemphasis{et al.} {[}\hyperlink{cite.introduction:id9}{ZKS20}{]} and Zhou and Koutsopoulos {[}\hyperlink{cite.introduction:id11}{ZK22}{]}) demonstrate how MIT RailSim has helped agencies address congestion, improve service reliability, and support long\sphinxhyphen{}term planning decisions.


\section{Features}
\label{\detokenize{introduction:features}}

\section{License}
\label{\detokenize{introduction:license}}
\sphinxAtStartPar
The project License is yet to be determined. Please contact the development team for more information.

\sphinxstepscope


\chapter{Setup Guide}
\label{\detokenize{setup_guide:setup-guide}}\label{\detokenize{setup_guide::doc}}
\sphinxAtStartPar
This project uses Pipenv for dependency management and packaging. You can learn about the dependencies looking at the Pipfile at the project root directory. Assuming you have a working version of pip, you can install Pipenv using pip:

\begin{sphinxVerbatim}[commandchars=\\\{\}]
\PYGZdl{}\PYG{+w}{ }pip\PYG{+w}{ }install\PYG{+w}{ }\PYGZhy{}\PYGZhy{}user\PYG{+w}{ }pipenv
\end{sphinxVerbatim}

\sphinxAtStartPar
Pipenv will take care of installing dependencies and creating the virtual environment. Run the following command from the project root directory (where the Pipfile and Pipfile.lock are located):

\begin{sphinxVerbatim}[commandchars=\\\{\}]
\PYGZdl{}\PYG{+w}{ }pipenv\PYG{+w}{ }install
\end{sphinxVerbatim}

\sphinxAtStartPar
You can activate the virtual environment using:

\begin{sphinxVerbatim}[commandchars=\\\{\}]
\PYGZdl{}\PYG{+w}{ }pipenv\PYG{+w}{ }shell
\end{sphinxVerbatim}

\sphinxAtStartPar
Depending on Python versions available on your system you may get an error like:

\begin{sphinxVerbatim}[commandchars=\\\{\}]
Error:\PYG{+w}{ }the\PYG{+w}{ }specified\PYG{+w}{ }Python\PYG{+w}{ }version\PYG{+w}{ }\PYG{o}{(}\PYG{l+m}{3}.8\PYG{o}{)}\PYG{+w}{ }is\PYG{+w}{ }not\PYG{+w}{ }available\PYG{+w}{ }on\PYG{+w}{ }your\PYG{+w}{ }system.
\end{sphinxVerbatim}

\sphinxAtStartPar
It is recommended to install the required Python version using pyenv. You can install pyenv using the instructions at \sphinxurl{https://github.com/pyenv/pyenv?tab=readme-ov-file\#installation}.

\sphinxAtStartPar
You may need to reactivate your shell after installing pyenv/pipenv to make sure they are added to the PATH.

\sphinxAtStartPar
Once you have pyenv installed, Pipenv will ask you if you like to use pyenv to install the required Python version. You can say yes and Pipenv will install the required Python version in the virtual environment.

\sphinxstepscope


\chapter{Getting Started}
\label{\detokenize{getting_started:getting-started}}\label{\detokenize{getting_started::doc}}

\section{Installation Guide}
\label{\detokenize{getting_started:installation-guide}}
\sphinxAtStartPar
Provide step\sphinxhyphen{}by\sphinxhyphen{}step instructions for installing the project, including prerequisites.


\section{Quick Start}
\label{\detokenize{getting_started:quick-start}}
\sphinxAtStartPar
A quick guide to getting a simple example up and running.

\sphinxstepscope


\chapter{Usage}
\label{\detokenize{usage:usage}}\label{\detokenize{usage::doc}}

\section{Basic Usage}
\label{\detokenize{usage:basic-usage}}
\sphinxAtStartPar
Explain how to use the project with simple examples.


\section{Advanced Usage}
\label{\detokenize{usage:advanced-usage}}
\sphinxAtStartPar
Dive into more complex use cases and features.

\sphinxstepscope


\chapter{Configuration}
\label{\detokenize{configuration:configuration}}\label{\detokenize{configuration::doc}}

\section{Configuration Options}
\label{\detokenize{configuration:configuration-options}}
\sphinxAtStartPar
Document configuration options, environment variables, etc.


\section{Default Configuration}
\label{\detokenize{configuration:default-configuration}}
\sphinxAtStartPar
Explain the default setup and how to customize it.

\sphinxstepscope


\chapter{Architecture}
\label{\detokenize{architecture:architecture}}\label{\detokenize{architecture::doc}}

\section{Components Overview}
\label{\detokenize{architecture:components-overview}}
\sphinxAtStartPar
Describe the main components of the project.


\section{Data Flow}
\label{\detokenize{architecture:data-flow}}
\sphinxAtStartPar
Explain how data flows through the system.

\sphinxstepscope


\chapter{Development}
\label{\detokenize{development:development}}\label{\detokenize{development::doc}}

\section{Development Environment Setup}
\label{\detokenize{development:development-environment-setup}}
\sphinxAtStartPar
Guide on setting up the development environment.


\section{Build Instructions}
\label{\detokenize{development:build-instructions}}
\sphinxAtStartPar
Explain how to build the project from source.


\section{Testing}
\label{\detokenize{development:testing}}
\sphinxAtStartPar
Describe how to run tests.

\sphinxstepscope


\chapter{Contribution Guidelines}
\label{\detokenize{contribution_guidelines:contribution-guidelines}}\label{\detokenize{contribution_guidelines::doc}}

\section{How to Contribute}
\label{\detokenize{contribution_guidelines:how-to-contribute}}
\sphinxAtStartPar
Instructions for making contributions, including coding standards and the pull request process.


\section{Community Guidelines}
\label{\detokenize{contribution_guidelines:community-guidelines}}
\sphinxAtStartPar
Code of conduct and how to get involved in the community.

\sphinxstepscope


\chapter{API Documentation}
\label{\detokenize{api_documentation:api-documentation}}\label{\detokenize{api_documentation::doc}}

\section{API Overview}
\label{\detokenize{api_documentation:api-overview}}
\sphinxAtStartPar
High\sphinxhyphen{}level overview of the API.


\section{Endpoints/Functions}
\label{\detokenize{api_documentation:endpoints-functions}}
\sphinxAtStartPar
Detailed descriptions of API endpoints/functions, including parameters, request/response formats, and examples.

\sphinxstepscope


\chapter{Faqs}
\label{\detokenize{faqs:faqs}}\label{\detokenize{faqs::doc}}
\sphinxAtStartPar
Address common questions and issues.

\sphinxstepscope


\chapter{Troubleshooting}
\label{\detokenize{troubleshooting:troubleshooting}}\label{\detokenize{troubleshooting::doc}}

\section{Common Issues}
\label{\detokenize{troubleshooting:common-issues}}
\sphinxAtStartPar
List common issues and their solutions.


\section{Getting Help}
\label{\detokenize{troubleshooting:getting-help}}
\sphinxAtStartPar
Information on where to ask questions or report issues.

\sphinxstepscope


\chapter{Changelog}
\label{\detokenize{changelog:changelog}}\label{\detokenize{changelog::doc}}

\section{Version History}
\label{\detokenize{changelog:version-history}}
\sphinxAtStartPar
List of changes for each version, including new features, bug fixes, and breaking changes.

\sphinxstepscope


\chapter{Appendix}
\label{\detokenize{appendix:appendix}}\label{\detokenize{appendix::doc}}

\section{Glossary}
\label{\detokenize{appendix:glossary}}
\sphinxAtStartPar
Definitions of terms used in the documentation.


\section{Further Reading}
\label{\detokenize{appendix:further-reading}}
\sphinxAtStartPar
Links to additional resources such as blog posts, tutorials, and papers.


\chapter{Indices and tables}
\label{\detokenize{index:indices-and-tables}}\begin{itemize}
\item {} 
\sphinxAtStartPar
\DUrole{xref,std,std-ref}{genindex}

\item {} 
\sphinxAtStartPar
\DUrole{xref,std,std-ref}{modindex}

\item {} 
\sphinxAtStartPar
\DUrole{xref,std,std-ref}{search}

\end{itemize}

\begin{sphinxthebibliography}{Zho22}
\bibitem[KW07]{introduction:id7}
\sphinxAtStartPar
Haris N. Koutsopoulos and Zhigao Wang. Simulation of Urban Rail Operations. \sphinxstyleemphasis{Transportation Research Record}, 2006:84\textendash{}91, 2007. URL: \sphinxurl{https://api.semanticscholar.org/CorpusID:110620690}.
\bibitem[WK11]{introduction:id8}
\sphinxAtStartPar
Zhigao Wang and Haris N. Koutsopoulos. Calibration of urban rail simulation models: a methodology using SPSA algorithm. In \sphinxstyleemphasis{Proceedings of the Winter Simulation Conference}, Wsc \textquotesingle{}11, 3704\textendash{}3714. Winter Simulation Conference, 2011.
\bibitem[Zho22]{introduction:id10}
\sphinxAtStartPar
Jiali Zhou. \sphinxstyleemphasis{Urban rail simulation and applications in service planning and operations}. PhD thesis, Northeastern University, 2022.
\bibitem[ZK22]{introduction:id11}
\sphinxAtStartPar
Jiali Zhou and Haris N. Koutsopoulos. Schedule\sphinxhyphen{}based Analysis of Transmission Risk in Public Transportation Systems. \sphinxstyleemphasis{ArXiv}, 2022. URL: \sphinxurl{https://api.semanticscholar.org/CorpusID:246904579}.
\bibitem[ZKS20]{introduction:id9}
\sphinxAtStartPar
Jiali Zhou, Haris N. Koutsopoulos, and Saeid Saidi. Evaluation of Subway Bottleneck Mitigation Strategies using Microscopic, Agent\sphinxhyphen{}Based Simulation. \sphinxstyleemphasis{Transportation Research Record}, 2674:649\textendash{}661, 2020. URL: \sphinxurl{https://api.semanticscholar.org/CorpusID:218922083}.
\end{sphinxthebibliography}



\renewcommand{\indexname}{Index}
\printindex
\end{document}